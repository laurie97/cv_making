\documentclass[]{letter}

\usepackage{geometry}
 \geometry{
 a4paper,
 right=25mm,
 left=25mm,
 top=10mm,
 }
\setlength\parindent{24pt}

\begin{document}
% If you want headings on subsequent pages,
% remove the ``%'' on the next line:
%\pagestyle{headings}

\begin{letter}{}  % Since good old William's 450th birthday in a few days ;-)
  \address{laurie.mcclymont@cern.ch\\
  +447951694231}
  \signature{Laurie McClymont}
  %\signature{Laurie M%\textsuperscript{c}Clymont}
  \opening{Dear Sir/Madam,}
  \vspace{1mm}

I am applying for the role of Data Analyst at Criteo, and I would like to explain why I am excited about this role and
how I believe I have the statistical, technical and communication skills required to succeed at Criteo.

The job interests me as I would be able to apply the analytic and technical skills learnt during my PhD
to create impact in a real world situation such as online advertising.
I relish the challenges of using cutting-edge statistical and computational techniques on complex and large data-sets to be able
to improve the effectiveness of advertising campaigns.
%and the opportunity to broaden my knowledge of the advertising market and those of your clients.
Finally, I know from my experience of working for 18 months at the CERN campus in Geneva,
that the international environment of Criteo within the diverse city of Paris would suit me well.

I have good analytic and statistical skills that would make me well qualified for this role.
My current research involves searching for signs of new physics in the large data sets from the LHC experiment,
leading to 3 public results in the last 18 months.
This research entails searching for 100s of discrepant points in a data-set of 10 million events;
this has required the use of a machine-learning algorithm, regression techniques, and  Bayesian hypothesis testing.
In addition I have experience and interest in statistical analysis beyond my current research;
my undergraduate physics degree required complex mathematical and statistical problem solving
and I completed statistical courses at UCL and at an international particle physics summer school.
In addition  I attended a CERN academic tutorial series on Machine Learning,
consisting of lectures from the experts in the field from CERN and from Google Brain.

To perform this research I have developed a range of computing skills that are essential for this role.
I have 3 years experience using computational tools to perform large-scale analyses using tools such as
Python within a collaborative environment.
Whilst working on these projects I have required to learn new computational skills quickly;
for example I began working on a new project where all of the existing framework used Python,
of which I had zero experience.
However, from studying the existing code and using online manuals
I was soon able to use and adapt the existing code to perform the analysis.
I believe that this ability to adapt to a new working environment gives me confidence that I would be able to fit in within Criteo.
I am also keen to broaden my current skill set;
to this end I have taught myself some basic usage of SQL and Python libraries such as seaborn, pandas and skicit-learn.

%To be able to analyse these large data-sets within the collaborative environment I have developed a range
%of computing tools ranging from expertise in the basic languages,
%to the specific computing tools used within our experiments software;
%showing that I have the ability to learn new computing skills
%allowing me to rapidly adapt to a different projects and working environments.

An equally important facet of my research experience is the communication skills.
I am a member of the ATLAS experiment, an international collaboration of 3,000 scientists,
performing a large array of scientific research.
To effectively carry out these tasks it is necessary to work together in small groups
with regular formal and informal communication within and between these groups.
I work well within this environment;
and as a result 6 months ago I was asked to be an analysis contact,
a role which involves co-ordinating the work of a small team and representing our work and needs to those outside of the team.

I have also been invited to give a number of presentations in a large range of situations;
from international scientific conferences to explaining particle physics to schools pupils.
From this I have learnt  that when presenting information it is essential to understand the audience that you are presenting to
and to think about what the key messages your audience should take away;
which allows for effective communication of your hard work.

Thank you for considering my application and I look forward to hearing from you

\vspace{5mm}

\closing{Yours faithfully}


\end{letter}
\end{document}

