%----------------------------------------------------------------------------------------
%	PACKAGES AND OTHER DOCUMENT CONFIGURATIONS
%----------------------------------------------------------------------------------------

\documentclass[10pt]{article} % Default font size
%%%%%%%%%%%%%%%%%%%%%%%%%%%%%%%%%%%%%%%%%
% Wilson Resume/CV
% Structure Specification File
% Version 1.0 (22/1/2015)
%
% This file has been downloaded from:
% http://www.LaTeXTemplates.com
%
% License:
% CC BY-NC-SA 3.0 (http://creativecommons.org/licenses/by-nc-sa/3.0/)
%
%%%%%%%%%%%%%%%%%%%%%%%%%%%%%%%%%%%%%%%%%

%----------------------------------------------------------------------------------------
%	PACKAGES AND OTHER DOCUMENT CONFIGURATIONS
%----------------------------------------------------------------------------------------

\usepackage[a4paper, hmargin=20mm, vmargin=5mm, top=5mm]{geometry} % Use A4 paper and set margins

\usepackage{fancyhdr} % Customize the header and footer

\usepackage{lastpage} % Required for calculating the number of pages in the document

\usepackage{hyperref} % Colors for links, text and headings


\usepackage{enumitem}
\setlist{nosep} % or \setlist{noitemsep} to leave space around whole list

\setcounter{secnumdepth}{0} % Suppress section numbering

%\usepackage[proportional,scaled=1.064]{erewhon} % Use the Erewhon font
%\usepackage[erewhon,vvarbb,bigdelims]{newtxmath} % Use the Erewhon font
\usepackage[utf8]{inputenc} % Required for inputting international characters
\usepackage[T1]{fontenc} % Output font encoding for international characters

%\usepackage{fontspec} % Required for specification of custom fonts
%\setmainfont[Path = ./fonts/,
%Extension = .otf,
%BoldFont = Erewhon-Bold,
%ItalicFont = Erewhon-Italic,
%BoldItalicFont = Erewhon-BoldItalic,
%SmallCapsFeatures = {Letters = SmallCaps}
%]{Erewhon-Regular}

\usepackage{color} % Required for custom colors
\definecolor{slateblue}{rgb}{0.17,0.22,0.34}

\usepackage{sectsty} % Allows customization of titles
\sectionfont{\color{slateblue}\fontsize{11}{11}\selectfont} % Color section titles

%\usepackage{fontspec}
\usepackage{lipsum}
\usepackage[explicit]{titlesec}
\titlespacing{\paragraph}{0pt}{0pt}{0pt}
\titlespacing\section{0pt}{12pt plus 2pt minus 2pt}{5pt plus 2pt minus 2pt}


\fancypagestyle{plain}{\fancyhf{}\cfoot{\thepage\ of \pageref{LastPage}}} % Define a custom page style
\fancypagestyle{plain}{\fancyhf{}} % Define a custom page style, without page number
\pagestyle{plain} % Use the custom page style through the document
\renewcommand{\headrulewidth}{0pt} % Disable the default header rule
\renewcommand{\footrulewidth}{0pt} % Disable the default footer rule

\setlength\parindent{0pt} % Stop paragraph indentation

% Non-indenting itemize
\newenvironment{itemize-noindent}
{\setlength{\leftmargini}{0em}
\begin{itemize}\setlength\itemsep{0em}}
{\end{itemize}}

% Text width for tabbing environments
\newlength{\smallertextwidth}
\setlength{\smallertextwidth}{\textwidth}
\addtolength{\smallertextwidth}{-0.5cm}

\newcommand{\sqbullet}{~\vrule height 1ex width .8ex depth -.2ex} % Custom square bullet point definition
\pagenumbering{gobble}

%----------------------------------------------------------------------------------------
%	MAIN HEADER COMMAND
%----------------------------------------------------------------------------------------

\renewcommand{\title}[1]{
{\LARGE{\color{slateblue}\textbf{#1}}}\\ % Header section name and color
}

%----------------------------------------------------------------------------------------
%	JOB COMMAND
%----------------------------------------------------------------------------------------

\newcommand{\job}[6]{
\begin{tabbing}
\hspace{2cm} \= \kill
#1 \> \href{#4}{\textbf{#3}} \\
#2 \>\+ \textit{#5} \\
\hspace{-1.5cm}
\begin{minipage}{\smallertextwidth}
\vspace{2mm}
#6
\end{minipage}
\end{tabbing}
}

\newcommand{\edu}[4]{
\begin{tabbing}
\hspace{2cm} \= \kill
{#1} \>\+ \href{#3}{\textbf{#2}}\\
\hspace{-1.5cm}
\begin{minipage}{\smallertextwidth}
\vspace{1mm}
#4
\end{minipage}
\end{tabbing}
}


%----------------------------------------------------------------------------------------
%	SKILL GROUP COMMAND
%----------------------------------------------------------------------------------------

\newcommand{\skill}[2]{
\textbullet \>  \textbf{#1} \> #2 \\
}

%----------------------------------------------------------------------------------------
%	INTERESTS GROUP COMMAND
%-----------------------------------------------------------------------------------------

\newcommand{\tabgroup}[1]{
\begin{tabbing}
\hspace{6mm} \= \kill
#1
\end{tabbing}
\vspace{-10mm}
}

\newcommand{\interest}[1]{\sqbullet \textbf{#1}\\[3pt]} % Define a custom command for individual interests

%----------------------------------------------------------------------------------------
%	TABBED BLOCK COMMAND
%----------------------------------------------------------------------------------------

\newcommand{\tabbedblock}[1]{
\begin{tabbing}
\hspace{2cm} \= \hspace{4cm} \= \kill
#1
\end{tabbing}
}
 % Include the file specifying document layout

\begin{document}

%----------------------------------------------------------------------------------------
%	NAME AND CONTACT INFORMATION
%----------------------------------------------------------------------------------------


\parbox{0.5\textwidth}{ % First block
\title{\center{Laurie M$^{c}$Clymont}} % Print the main header
}
\hspace{2cm} % Horizontal space between the two blocks
\parbox{0.5\textwidth}{ % Second block  
  \begin{tabbing} % Enables tabbing 
    \hspace{2cm} \= \hspace{3cm} \= \kill % Spacing within the block
           {\bf Email} \> \href{mailto:laurie.mcclymont@cern.ch}{laurie.mcclymont@cern.ch} \\ % Email address
           {\bf Mobile} \> +447951694231  \\ % Mobile phone
           {\bf LinkedIn} \> \href{https://www.linkedin.com/in/laurie-mcclymont-695520118/}{Laurie-McClymont} % LinkedIn
  \end{tabbing}
}

\rule{\textwidth}{0.5mm}

%----------------------------------------------------------------------------------------
%	PERSONAL PROFILE
%----------------------------------------------------------------------------------------

\vspace{-2mm}
\section{Personal Profile}
\begin{itemize}
\item{UCL physics PhD student, currently analysing large data sets at the LHC experiment at CERN.}
\item{Experienced in using computational tools, such as Python, to analyse and understand large data sets.}
\item{Strong communication skills; including presenting complex information and collaborative working.}
\item{My research experience, natural curiosity and strong mathematical background make me well suited to a role
  analysing Facebook's unique data-set to drive growth at the world's most exciting enterprise.}
\end{itemize}


\vspace{0.2mm}
\rule{\textwidth}{0.5mm}



%----------------------------------------------------------------------------------------
%	TECHNICAL SKILLS SKILLS SECTION
%----------------------------------------------------------------------------------------
\vspace{-1mm}
\section{Relevant Skills}
\begin{tabbing}
  \hspace{3mm} \= \hspace{30mm} \= \kill 
  \skill{Computing}
        {
          Languages: Python (pandas, seaborn, scikit-learn), C++, bash. \\
          \>\>Platforms: Linux, Mac OSX. \\
          \>\>Others: Git, LaTeX, Excel, PowerPoint, Word, SQL (Basic).
        }
  \skill{Statistics}  { Regression, hypothesis testing, error analysis, machine learning, likelihoods. }
  \skill{Communication}
        {
          Selected to present at national and international scientific conferences.\\
          \>\>Strong team player; worked within large teams including in a leadership role.
        }
\end{tabbing}

%------------------------------------------------

%----------------------------------------------------------------------------------------
%	EMPLOYMENT HISTORY SECTION
%----------------------------------------------------------------------------------------
\vspace{-9mm}
\section{Relevant Experience}
\job
{Sep 2014 -}{Sep 2017}
{High Energy Physics Group, University College London}
{https://www.hep.ucl.ac.uk}
{PhD Candidate}
{
  \begin{itemize-noindent}
  \item{Member of the ATLAS experiment searching for new physics using large data sets.}
    \begin{itemize}
    \item{Worked within many diverse teams, in an international collaboration of 3,000 scientists.}
    \item{Performed large scale data analysis projects using Python, C++ and GitHub.}
    \item{Included an 18 month long-term attachment at the main CERN campus in Geneva.} 
    \end{itemize}
  \item{Lead analyser measuring the efficiency of the ATLAS $b$-jet trigger.}
    \begin{itemize}
    \item{Performed a technical measurement and error analysis that is essential for the experiment.}
    \item{Identified a critical problem within the data, and developed a strategy to successfully mitigate the issue.}
    \item{Completed the measurement to deadlines and effectively communicated results to collaborators.}
    \end{itemize}
  \item{Analysis contact for a team searching for new physics using pairs of $b$-jets.}
    \begin{itemize}
    \item{Applied statistical techniques including regression, machine learning and hypothesis testing.}
    \item{Appointed analysis contact; involved co-ordinating a team of five scientists,
      liaising with experts from other groups and reporting progress and plans to management.}
    \item{Published three public results in 2016; pushing the limits of our knowledge into unexplored regions.}
    %\item{Led the use of novel data aquisition techniques to extend the reach of the search.}
    \end{itemize}
 \item{Presented conclusions of data analysis to a range of audiences.}
   \begin{itemize}
   \item{Selected to summarise results to large scientific audiences at international conferences and workshops.}
   \item{Routinely reported details of analysis to technical meetings in ATLAS and at UCL.}
   \item{Presented current research to school pupils to inspire the next generation of scientists.}
   \end{itemize}
 \end{itemize-noindent}
}
%------------------------------------------------
\job
{June 2012 -}{Sep 2012}
{Institute of Astronomy, University of Cambridge}
{http://www.ast.cam.ac.uk}
{Summer Research Intern}
{\begin{itemize-noindent}
  \item{Spent eight weeks during the summer analysing data from two large astronomical telescopes.}
  \item{Used a statistical profile likelihood method to identify possible ``quasar'' candidates for further study.}
\end{itemize-noindent}}

\job
{Oct 2010 -}{Oct 2011}
{Merton College, University of Oxford}
{}
{Student Access Representative}
{\begin{itemize-noindent}
  \item{Engaged with students from disadvantaged backgrounds to inspire and encourage applications to university.}
  \item{Led a group of students to create a prospectus, advertising the college from a student's perspective.}
 \end{itemize-noindent}}

%----------------------------------------------------------------------------------------
%	EDUCATION SECTION
%----------------------------------------------------------------------------------------

\section{Education}
\edu
{2010-14}
{Merton College, University of Oxford}
{https://www.merton.ox.ac.uk/node/111}
{
  \begin{itemize-noindent}
  \item{MPhys Physics - 2:1 (68\%)}
  \item{Involved mathematical and statistical problem solving for a range of situations.}
  \end{itemize-noindent}
}

\edu
{2004-10}
{Altrincham Grammar School For Boys}
{https://www.agsb.co.uk}
{
  \begin{itemize-noindent}
  \item{A-Levels: Maths, Further Maths, Physics, History (A*, A*, A*, B).}
  \end{itemize-noindent}
}

%------------------------------------------------



%----------------------------------------------------------------------------------------
%	INTERESTS SECTION
%----------------------------------------------------------------------------------------
\vspace{-5mm}
\section{Interests}

\begin{tabbing}
  \hspace{2mm} \= \hspace{18mm} \= \kill 
  \skill{Sports}   {Play regularly in a 6-a-side rugby and football team. Keen runner and cyclist.}
  \skill{French}   {Conversational level. Practice through weekly in person conversations with French natives for 2 years.}
  \skill{Travel}   {Enjoy exploring new cities, countries and their cultures.}
\end{tabbing}

\vspace{-4mm}
Referees available upon request


\end{document}

%----------------------------------------------------------------------------------------
%	END OF CV
%----------------------------------------------------------------------------------------

\newpage

%----------------------------------------------------------------------------------------
%	OTHER IDEAS
%----------------------------------------------------------------------------------------

\section{Ideas of Other Relevant Experience}

\job
{June 2013 -}{Sep 2013}
{Rutherford-Appleton Laboratory, Didcot}
{}
{Summer Research Intern}
{\begin{itemize-noindent}
  \item{Spent eight weeks analysing a particular new physics model using a C++ framework.}
  \item{Identified an angular variable could be used to separate signal from background.}
\end{itemize-noindent}}

\vspace{10mm}
Still to do...
\vspace{10mm}


\job
{October 2010 -}{October 2011}
{Private projects}
{Plan to do...}
{}
{\begin{itemize-noindent}
  \item{Used random-trees algorithm to predict survival of titanic passengers, using a tutorial}
  \item{More to follow}
 \end{itemize-noindent}}







%----------------------------------------------------------------------------------------
%	REFEREE SECTION
%----------------------------------------------------------------------------------------

\section{Referees}

\parbox{0.5\textwidth}{ % First block
\begin{tabbing}
\hspace{2.75cm} \= \hspace{4cm} \= \kill % Spacing within the block
{\bf Name} \> Bill Lumbergh \\ % Referee name
{\bf Company} \> Initech Inc. \\ % Referee company
{\bf Position} \> Vice President \\ % Referee job title 
{\bf Contact} \> \href{mailto:bill@initech.com}{bill@initech.com} % Referee contact information
\end{tabbing}}
\hfill % Horizontal space between the two blocks
\parbox{0.5\textwidth}{ % Second block
\begin{tabbing}
\hspace{2.75cm} \= \hspace{4cm} \= \kill % Spacing within the block
{\bf Name} \> Michael "Big Mike" Tucker\\ % Referee name
{\bf Company} \> Burbank Buy More \\ % Referee company
{\bf Position} \> Store Manager \\ % Referee job title 
{\bf Contact} \> \href{mailto:mike@buymore.com}{mike@buymore.com} % Referee contact information
\end{tabbing}
}

%----------------------------------------------------------------------------------------

